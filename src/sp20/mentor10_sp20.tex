\documentclass[11pt]{exam}
\usepackage{commonheader}

% Use the following to toggle solutions/metas
\printsolutions % uncomment me
%\printmeta % uncomment me

\discnumber{10}
\title{Heaps \& Hashing}
\date{March 30, 2020}

\begin{document}
\maketitle
\begin{solution}
Link to meta: https://tinyurl.com/vbkl5dp
\end{solution}

\section{Heaps Review}
The following invariants apply for heaps: 
\begin{itemize}
  \item For a max-heap: for every node, the value of both its children must be less than the value of itself. 
  \item For a min-heap: for every node, the value of both its children must be greater than the value of itself. 
\end{itemize}
Heaps are usually represented using arrays, and can be created through heapification. Heapification can either be done level order + bubbling up (O(NlogN)), or reverse level order + bubbling down (O(N)). 
\begin{questions}
\subimport{topics/heaps/easy/}{min-heapify-example.tex}
\marginpar{\subimport{topics/heaps/}{example-tree2.tex}}
\end{questions}

% \section{More Heaps}
% \begin{questions}
% \subimport{../topics/heaps/easy/}{min-heap-sorted-array.tex}
% \subimport{../topics/heaps/easy/}{sorted-array-max-heap.tex}
% \end{questions}
\section{Min-Heapify This}
\begin{questions}
\subimport{topics/heaps/easy/}{methods.tex}
\subimport{topics/heaps/easy/}{min-heap-sorted-array.tex}
\subimport{topics/heaps/easy/}{sorted-array-max-heap.tex}
\end{questions}

\section{K Largest Items}
\begin{questions}
\subimport{topics/heaps/medium/}{k-largest-items.tex}
\end{questions}

\section{Ls for LinkedLists}
\begin{questions}
\subimport{topics/hashing/easy/}{motivation.tex}
\end{questions}

\section{Hashing Practice}
\begin{questions}
\subimport{topics/hashing/basics/}{diagram.tex}
\end{questions}
\clearpage

\section{Kelp!}
\begin{questions}
\subimport{topics/heaps/hard/}{sixty-one-pq.tex}
\end{questions}

\section{\extra{Game Trees}}
\begin{questions}
\subimport{topics/trees/game-trees/}{game-tree-easy-2.tex}
\end{questions}

\section{\extra{Hash Codes}}
\begin{questions}
\subimport{topics/hashing/hash-codes/}{intro.tex}
\subimport{topics/hashing/hash-codes/}{person.tex}
\subimport{topics/hashing/hash-codes/}{phonebook.tex}
\clearpage
\subimport{topics/hashing/hash-codes/}{poketime.tex}
\end{questions}


\end{document}
