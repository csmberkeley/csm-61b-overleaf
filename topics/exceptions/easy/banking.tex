%\begin{blocksection}
\question
\begin{parts}

\item The CSM content mentors have opened a bank called CSM Bank! This bank allows customers to store money and take out loans.

The bank keeps track of 10 customers, where each customer is represented by a number from 0 to 9 (inclusive). A customer of the bank can either deposit some amount into their account, withdraw some amount from their account,
take out a loan, or pay back a loan using their account number \lstinline{account_number}.

However, to comply with UC Berkeley regulations, CSM needs your help to make the system just a bit more responsible and secure! 
First, a customer cannot withdraw money from the bank if it causes their account to be negative. Also, a customer should not be able to take out a loan if they already owe the bank money.

In addition, the bank keeps track of two other values for regulatory reasons: \lstinline{total_money} represents the amount of physical money the bank has in its vaults, 
and \lstinline{total_assets} represents the amount of money the bank is responsible for, i.e., the total amount of money in the bank's accounts.
In order to ensure sufficient reserves, the bank should never give out a loan if giving the loan causes the amount of money in the bank's vaults to fall below 20\% of its total assets.

Please implement exceptions in the methods below to ensure that CSM Bank complies with UC Berkeley regulations.

\ifprintanswers\else
\begin{lstlisting}

class Bank {

    double total_money;
    double total_assets;
    double[] accounts;
    double[] loans;
    
    public Bank() {
        total_money = 0;
        total_assets = 0;
        accounts = {0, 0, 0, 0, 0, 0, 0, 0, 0, 0};
        loans = {0, 0, 0, 0, 0, 0, 0, 0, 0, 0};
    }
    
\end{lstlisting}

\pagebreak
\begin{lstlisting}
 
    public double deposit(int account_number, double amount) {







        total_money += amount
        total_assets += amount
        accounts[account_number] += amount;
        return accounts[account_number];
    }
    
\end{lstlisting}
\begin{lstlisting}
 
    public double withdraw(int account_number, double amount) {







        total_money -= amount;
        total_assets -= amount;
        accounts[account_number] -= amount;
        return accounts[account_number];
    }
    
\end{lstlisting}
\begin{lstlisting}
 
    public double take_loan(int account_number, double amount) {







        total_money -= amount;
        loans[account_number] += amount;
        return loans[account_number];
    }
    
\end{lstlisting}
\pagebreak
\begin{lstlisting}
 
    public double pay_loan(int account_number, double amount) {







        total_money += amount;
        loans[account_number] -= amount;
        return loans[account_number];
    }
}

\end{lstlisting}
\fi

\begin{solution}
\begin{lstlisting}
class Bank {

    double total_money;
    double total_assets;
    double[] accounts;
    double[] loans;
    
    public Bank() {
        total_money = 0;
        total_assets = 0;
        accounts = {0, 0, 0, 0, 0, 0, 0, 0, 0, 0};
        loans = {0, 0, 0, 0, 0, 0, 0, 0, 0, 0};
    }

    public double deposit(int account_number, double amount) {
        if (account_number > 9 || account_number < 0) {
            throw IllegalArgumentException("Please enter a valid account number");
        }
        
        total_money += amount
        total_assets += amount
        accounts[account_number] += amount;
        return accounts[account_number];
    }

    public double withdraw(int account_number, double amount) {
        if (account_number > 9 || account_number < 0) {
            throw IllegalArgumentException("Please enter a valid account number");
        }
        if (amount > this.accounts[account_number]) {
            throw Exception("You do not have the necessary funds to make a withdrawal");
        }
        
        total_money -= amount;
        total_assets -= amount;
        accounts[account_number] -= amount;
        return accounts[account_number];
    }

    public double take_loan(int account_number, double amount) {
        if (account_number > 9 || account_number < 0) {
            throw IllegalArgumentException("Please enter a valid account number");
        }
        if (this.total_money - amount < 0.2 * this.total_assets) {
            throw Exception("We do not have the funds to supply this loan");
        }
        
        total_money -= amount;
        loans[account_number] += amount;
        return loans[account_number];
    }

    public double pay_loan(int account_number, double amount) {
        if (account_number > 9 || account_number < 0) {
            throw IllegalArgumentException("Please enter a valid account number");
        }
        if (loans[account_number] < 0) {
            throw Exception("Pay us back your loan first!");
        }
        
        total_money += amount;
        loans[account_number] -= amount;
        return loans[account_number];
    }
}

\end{lstlisting}
\end{solution}
    
\end{parts}

%\end{blocksection}
