%\begin{blocksection}
\question
\begin{lstlisting}[aboveskip=-\baselineskip]


try {
    doSomething();
} catch (ArrayIndexOutOfBoundsException e) {
    System.out.println("caught array index exception");
} catch (Exception e) {
    System.out.println("caught an exception");
    throw e;
} catch (NullPointerException e) {
    System.out.println("caught null pointer exception");
} finally {
    System.out.println("may be an ID-10T error, consult 61B content mentors to check.");
}
\end{lstlisting}

\begin{parts}
\part This code causes a compile time error. Which blocks could we remove to fix the issue?

\begin{solution}
The \lstinline$catch (NullPointerException e)$ causes a compile time error because it is unreachable, 
because \lstinline$catch (Exception e)$ will always catch null pointer errors before we reach the line \lstinline$catch (NullPointerException e)$. 

Java doesn't like ambiguity or unreachable code, so it causes a compile time error. (Q: Why would a NullPointerException count as an Exception? A: Because NullPointerException is a subclass of Exception!) 

We can remove either \lstinline$catch (NullPointerException e)$ or \lstinline$catch (Exception e)$ to fix the error. Discuss with your students the pros and cons of removing either, then see if anyone has suggestions for a way we could fix the code without removing either line. (Swap the positions of the block \lstinline$catch (Exception e)$ and \lstinline$catch (NullPointerException e)$.)
\end{solution}

\clearpage

\part We have fixed the problematic code causing the compile time error. What will now print if \lstinline$doSomething()$ throws a
\lstinline$NullPointerException$?
\begin{solution}[0.75in]
Students will have different answers depending on how they chose to fix the code. If they removed \lstinline$catch (NullPointerException e)$, the correct answer would be:
\begin{verbatim}
caught an exception
may be an ID-10T error, consult 61B content mentors to check.
NullPointerException
\end{verbatim}
If they removed \lstinline$catch (Exception e)$ or swapped the location of the blocks, the answer would be:
\begin{verbatim}
caught null pointer exception
may be an ID-10T error, consult 61B content mentors to check.
\end{verbatim}
If time allows, it might be worthwhile to discuss all possibilities here.
Point out that \lstinline$NullPointerException$ is on the last line of output for the first solution but not the second because of the \lstinline$throw e$ line in the second block of the original code.
\end{solution}

\part What if \lstinline$doSomething()$ throws an
\lstinline$ArrayIndexOutOfBoundsException$?
\begin{solution}[0.75in]
\vspace{-2em}
\begin{verbatim}
caught array index exception
may be an ID-10T error, consult 61B content mentors to check.
\end{verbatim}
\end{solution}

\part What if \lstinline$doSomething()$ doesn't error?
\begin{solution}[0.75in]
\begin{verbatim}
may be an ID-10T error, consult 61B content mentors to check.
\end{verbatim}
\end{solution}
\end{parts}

\begin{solution}
\textbf{Meta:} Remember that completing a catch prevents all other catches
in the same block from running. Even on a re-throw, no other catches in the
same block are called, because the exception has already been `handled'!
\end{solution}
%\end{blocksection}
