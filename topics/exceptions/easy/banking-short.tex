%ID: 61222
\question
\begin{parts}

\item The CSM content mentors have opened a bank called CSM Bank!

The bank keeps track of 10 customers, where each customer is represented by a number from 0 to 9 (inclusive). A bank customer can either deposit some amount into their account or withdraw some amount from their account using their account number \lstinline{account_number}.

However, to comply with UC Berkeley regulations, CSM needs your help to make the system just a bit more responsible and secure!
Of course, a customer cannot withdraw money from the bank if it causes their account to be negative.
We'd also like to make sure arguments to \lstinline{deposit} and \lstinline{withdraw} are valid; think about what values of \lstinline{account_number} and \lstinline{amount} we wouldn't allow for each method.

Please implement exceptions in the methods below to ensure that CSM Bank complies with UC Berkeley regulations.
Don't forget to include informative error messages!
The following built-in Java exceptions may be helpful:
\begin{itemize}
    \item \lstinline{IllegalArgumentException}: Method parameter is a bad input. Method cannot be executed with given inputs.
    \item \lstinline{IndexOutOfBoundsException}: Some index is out of bounds.
\end{itemize}

\ifprintanswers\else
\begin{lstlisting}

class Bank {

    double total_money;
    double[] accounts;
    
    public Bank() {
        total_money = 0;
        accounts = {0, 0, 0, 0, 0, 0, 0, 0, 0, 0};
    }
    
\end{lstlisting}

\pagebreak
\begin{lstlisting}
 
    public double deposit(int account_number, double amount) {
        if (__________________________________) {
            throw new _________________________(____________________________________);
        }
        
        if (__________________________________) {
            throw new _________________________(____________________________________);
        }

        total_money += amount;
        accounts[account_number] += amount;
        return accounts[account_number];
    }
 
    public double withdraw(int account_number, double amount) {
        if (__________________________________) {
            throw new _________________________(____________________________________);
        }
        
        if (__________________________________) {
            throw new _________________________(____________________________________);
        }
        
        if (__________________________________) {
            throw new _________________________(____________________________________);
        }

        total_money -= amount;
        accounts[account_number] -= amount;
        return accounts[account_number];
    }
}
\end{lstlisting}
\fi

\begin{solution}
\begin{lstlisting}
class Bank {

    double total_money;
    double[] accounts;
    
    public Bank() {
        total_money = 0;
        accounts = {0, 0, 0, 0, 0, 0, 0, 0, 0, 0};
    }

    public double deposit(int account_number, double amount) {
        if (account_number > 9 || account_number < 0) {
            throw new IndexOutOfBoundsException("Please enter a valid account number");
        }
        if (amount <= 0) {
            throw new IllegalArgumentException("Please enter a positive amount to deposit");
        }
        
        total_money += amount;
        accounts[account_number] += amount;
        return accounts[account_number];
    }

    public double withdraw(int account_number, double amount) {
        if (account_number > 9 || account_number < 0) {
            throw new IndexOutOfBoundsException("Please enter a valid account number");
        }
        if (amount <= 0) {
            throw new IllegalArgumentException("Please enter a positive amount to withdraw");
        }
        if (amount > this.accounts[account_number]) {
            throw new IllegalArgumentException("You do not have the necessary funds to make a withdrawal");
        }
        
        total_money -= amount;
        accounts[account_number] -= amount;
        return accounts[account_number];
    }
}
\end{lstlisting}
Note that we do not need to check for null values since \lstinline{null} cannot be assigned to Java primitives.
\end{solution}
    
\end{parts}