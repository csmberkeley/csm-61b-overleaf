%61226
\begin{blocksection}
\question Implement \lstinline$fromSortedArray$ for binary search trees. Given
a sorted \lstinline$int[] array$, efficiently construct a balanced binary
search tree containing every element of the array. For an efficient implementation,
try to minimize the runtime of \lstinline$fromSortedArray$ while ensuring
that the final BST is balanced.

\ifprintanswers
\else
\begin{lstlisting}
public class BinarySearchTree<T extends Comparable<T>> {
    protected Node root;
    protected class Node {
        public T value;
        public Node left;
        public Node right;
    }
    public static BinarySearchTree<Integer> fromSortedArray(int[] values) {
        BinarySearchTree<Integer> bst = new BinarySearchTree<>();
        bst.root = bst.fromSortedArray(values, 0, values.length - 1);
        return bst;
    }
    private Node fromSortedArray(int[] values, int lower, int upper) {














\end{lstlisting}
\fi

\begin{solution}
\begin{lstlisting}
public class BinarySearchTree<T extends Comparable<T>> {
    protected Node root;
    protected class Node {
        public T value;
        public Node left;
        public Node right;
    }
    public static BinarySearchTree<Integer> fromSortedArray(int[] values) {
        BinarySearchTree<Integer> bst = new BinarySearchTree<>();
        bst.root = bst.fromSortedArray(values, 0, values.length - 1); // setting a new root
        return bst;
    }
    private Node fromSortedArray(int[] values, int lower, int upper) {
       if (lower > upper) {
           return null;
       }
       int middle = lower + ((upper - lower) / 2); // middle index of the array
       Node mid = new Node();
       mid.value = values[middle];
       mid.left = fromSortedArray(values, lower, middle - 1);  // recurse on the left half
       mid.right = fromSortedArray(values, middle + 1, upper); // recurse on the right half
       return mid;
    }
}
\end{lstlisting}

A binary tree is any tree where each node has at most two children.
A binary search tree enforces the additional constraint
that, given any parent node, nodes within the left subtree are smaller in value than the parent's value and the
nodes within the right subtree are larger in value than the parent's value.

The main idea behind this solution is that, given an array of integers, in order to minimize the
height of a binary search tree, you'd want to construct your BST with the median of the array as
your root. This way, you "evenly distribute" nodes among the left and right subtrees. By extending
this insight recursively, we can construct a completely balanced BST.

As an interesting side note, the reason why we use \lstinline{int middle = lower + ((upper - lower) / 2);} instead
of \lstinline{int middle = (upper + lower) / 2;} is to help prevent buffer overflows for large lists.
\end{solution}
\end{blocksection}
