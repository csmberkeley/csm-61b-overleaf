%%%%%%%%%%%%%Oscar Ortega - Tutor Spring 5/8/2019%%%%%%%
%ID: 61201
\question Those darned Stanford Students are at it again! In an attempt to reverse-engineer your code, they stole your \lstinline$get$ and \lstinline$put$ methods! Fortunately, you understand red-black trees pretty well. Consider your now incomplete \lstinline$RBTree$ implementation.
\begin{lstlisting}

public class RBTree<Key extends Comparable<Key>, Value> {
    private static final boolean RED   = true;
    private static final boolean BLACK = false;
    private Node root;     // root of the BST

    private class Node {
        private Key key;           // key
        private Value val;         // associated data
        private Node left, right;  // links to left and right subtrees
        private boolean color;     // color of parent link
        private int size;          // subtree count
    
        public Node(Key key, Value val, boolean color, int size) {
            this.key = key;
            this.val = val;
            this.color = color;
        }
    }
    
    public RBTree() {...}
    private Node rotateRight(Node h) {...}
    private Node rotateLeft(Node h){...}
    private void flipColors(Node h) {...}
    private boolean isRed(Node h) {...}  // returns false if node is black or null
    private int size(Node h) {...}
    
    public void put(Key key, Value val) {
        root = put(root, key, val);
        root.color = BLACK;
    }


\end{lstlisting}
\newpage
\begin{enumerate}
   
\begin{lstlisting}
    private Value get(Key key) {
        Node runner = ________;
        while (___________) {
            int cmp = ______________;
            if (_______) {
                runner = __________;
            } else if (________) {
                runner = __________;
            } else {
                ________________;
            }
        }
        return null;
    }
\end{lstlisting}
\begin{solution}
\begin{lstlisting}
     private Value get(Key key) {
        Node runner = root;
        while (runner != null) {
            int cmp = key.compareTo(runner.key);
            if (cmp < 0) {
                runner = runner.left;
            } 
            else if (cmp > 0) {
                runner = runner.right;
            }
            else {
                return runner.val;
            }          
        }
        return null;
    }
\end{lstlisting}
Note how this is identical to how one would retrieve items in a regular binary search tree.
\end{solution}
\begin{lstlisting}
    private Node put(Node h, Key key, Value val) { 
        if (h == _____){
            return _____________;
        }
        int cmp = _____________;
        if  (_________) {
            h.left  = _____________; 
        } else if (__________) {
            h.right = _____________; 
        } else {
             h.val = _____;
        }          
        if (_____________ && _____________)) {
             h = rotateLeft(h);
        }    
        if (_____________  &&  _____________) {
            h = rotateRight(h);
        }
        if (_____________ && _____________) {
            _____________;
        }    
        h.size = _______________;
        
        return h;
    }
 }
\end{lstlisting}
\begin{solution}
\begin{lstlisting}
    private Node put(Node h, Key key, Value val) { 
        if (h == null){
            return new Node(key, val, RED, 1);
        }
        int cmp = key.compareTo(h.key);
        if  (cmp < 0) {
            h.left  = put(h.left,  key, val); 
        } else if (cmp > 0) {
            h.right = put(h.right, key, val); 
        } else {
             h.val = val;
        }          
        if (isRed(h.right) && !isRed(h.left)) {
             h = rotateLeft(h);
        }    
        if (isRed(h.left)  &&  isRed(h.left.left)) {
            h = rotateRight(h);
        }
        if (isRed(h.left)  &&  isRed(h.right)) {
            flipColors(h);
        }    
        h.size = size(h.left) + size(h.right) + 1;
        
        return h;
    }
\end{lstlisting}
Note how this simply sinks the current node down to its correct location in the now-potentially unbalanced tree before proceeding to correct issues with the current link colorings. Go through the different cases with your section if they are unfamiliar.
\end{solution}
\end{enumerate}
\clearpage