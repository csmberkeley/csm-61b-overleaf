%ID: 612069
% \usepackage{mathtools}
% \DeclarePairedDelimiter{\floor}{\lfloor}{\rfloor}

% \begin{blocksection}
\begin{parts}
\item For a node in a 2-3 tree with $n$ values, what are the possible numbers of children that each node can have? What if the node is not a leaf?

\begin{solution}[2in]
If $n = 1$, then the node can have 0 or 2 children. If $n = 2$, then the node can have 0 or 3 children. 

If the node is not a leaf, then if $n=1$, the node must have exactly 2 children, and if $n = 2$, the node must have exactly 3 children.
\end{solution}

\item Instead of splitting an overstuffed node by moving the middle value upwards in the tree, suppose we move the rightmost (third) value of a node. Can we still guarantee the idea of a \textit{balanced} binary search tree with 2-3 trees using this splitting method?

\begin{solution}[2in]
We cannot guarantee a balanced tree. For example, consider this initial 2-3 tree and sequence of insertions: 
\begin{center}
\begin{tikzpicture}[very thick,level/.style={sibling distance=20mm/#1},level distance=30pt]
\node[vertex, ellipse]{$1 \; 2$};
\end{tikzpicture}
\end{center}
\quad
If we were to insert 5, we would get: 
\begin{center}
\begin{tikzpicture}

\node[vertex]{$5$}
child{
    node[vertex, ellipse]{1 \; 2}};
\end{tikzpicture}
\end{center}
\quad
If we were to then insert 4 and 3, we would get:
\begin{center}
\begin{tikzpicture}
\node[vertex]{$5$}
child{
    node[vertex, ellipse]{3 \; 4}
    child{
        node[vertex,ellipse]{1 \; 2}}};
\end{tikzpicture}
\end{center}
which is not a balanced search tree.

On a more conceptual level, in order to minimize the height of our 2-3 tree, we choose to move the middle value upwards because doing so splits nodes more symmetrically. In other words, being the middle value of the node, we know that there exists values both less than and greater than it. So, we can naturally create both left and right branches of the newly promoted middle value.
\end{solution}

\item What is the maximum height of a 2-3 tree with $n$ values? (Hints: 
(1) Can a 2-3 tree have no overstuffed nodes? (2) Think about
what happens to the height of a tree when $n$ reaches a power of 2.)

\begin{solution}[2in]
Recall that a 2-3 tree is always balanced. In the worst case, our height is maximized if we have no overstuffed nodes (i.e., all nodes contain a single value). 
Hence, the maximum height of a 2-3 tree with $n$ values is equivalent to the height of a balanced BST with $n$ values.

It can be difficult to calculate the height of a balanced BST directly, but we can observe some initial patterns:
\begin{itemize}
    \item A balanced BST with height zero can have only $n=1$ element.
    \item A balanced BST with height 1 can have $n=2, 3$ elements.
    \item A balanced BST with height 2 can have $n=4, 5, 6, 7$ elements.
    \item $\cdots$
\end{itemize}
Note that when $n$ reaches a power of 2, the height of our balanced BST increases by 1. Hence, the maximum height of a 2-3 tree with $n$ nodes would be $\floor{\log_2(n)}$.
\end{solution}

\item Is it possible for leaf nodes of a 2-3 tree to have differing distances to the root? If not, explain why. Otherwise, provide an example 2-3 tree that satisfies the condition.

\begin{solution}%[2in]
No. Although our precise definition of "balance" checks that the maximum height difference between sub-trees is less than or equal to 1 (in addition to having both sub-trees be balanced), the 2-3-4 tree takes it a step even further by ensuring that each leaf node is the exact same distance to the root. This is because each value we insert into the tree either gets stuffed into a leaf node, moves upwards, and/or splits the node at the \textit{root} to increase the height of the overall tree -- thus, the branches of the tree never grow individually. In other words, whenever the distance from a leaf to the root increases, we know that this is because a value has been promoted at the root of the tree. And if a value at the root is promoted, then the distance from the root to \textit{every} leaf node is increased, so these distances will never be different.

\end{solution}

\end{parts}

% \end{blocksection}