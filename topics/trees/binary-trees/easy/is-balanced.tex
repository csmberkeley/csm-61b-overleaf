\begin{blocksection}
\question Define a procedure, \lstinline$isBalanced$, which takes a
\lstinline$Node$ and outputs whether or not the tree is balanced. A tree is
\define{balanced} if the left and right branches differ in height by at most
one and are themselves balanced. 

Hint: You can use height function defined above. 

\ifprintanswers
\else
\begin{lstlisting}
private boolean isBalanced(Node node) {









}
\end{lstlisting}
\fi

\begin{solution}
\begin{lstlisting}
private boolean isBalanced(Node node) {
    if (node == null) {
        return true;
    } else if (Math.abs(height(node.left) - height(node.right)) <= 1) {
        return isBalanced(node.left) && isBalanced(node.right);
    }
    return false;
}
\end{lstlisting}
\end{solution}

What is the runtime of \lstinline$isBalanced$?
\begin{solution}[0.25in]
$\Theta(N)$ in the best case, $\Theta(N \log N)$ in the worst case.  This can
also be read as $\Omega(N), O(N \log N)$ overall.

The best case is if the tree is unbalanced at the root, meaning that the
difference in the height of the root's left branch and the root's right branch
is greater than one. In this case, we just call \lstinline$height$ twice, once
on the left branch and once on the right subtree. After these two calls, we can
immediately see that the tree is unbalanced, so we return false. This leads to
a runtime of $\Theta(N)$ in the best case.

The worst case is if the tree is perfectly balanced. In this case, first, we will call \lstinline$height$ 
on \lstinline$node.left$ and \lstinline$node.right$. Each of these nodes has a sub-tree of roughly $N/2$ nodes, 
and so at this level, 2 height calls are made, each of which costs $N/2$. The total work done on this level is
$\Theta(N)$. Next, \lstinline$node.left$ will call \lstinline$height$ on its left and right children, and \lstinline$node.right$ will do the same. These children are now on the third level of the tree (the root node being the first level). Note that these third-level children now have a subtree of roughly $N/4$ nodes each. 4 \lstinline$height$ calls are made at this level, for a total cost of $\Theta(N)$ at this level too. As we keep going, each level will do $\Theta(N)$ work. Note that the bottom-most level (leaf-level) of such a perfectly balanced tree would have roughly $N/2$ nodes, and each height call would take constant time, for a total of $\Theta(N/2) = \Theta(N)$ work at the leaf-level too. 

Now that we have established that each level does $\Theta(N)$ work, all that we need to figure out is how many levels
there are in our worst case situation. This tree has $\log n$ levels, since a perfectly balanced tree has $\log n$ levels. Therefore, the total runtime cost for the worst case is $\Theta(N \log N)$, which can also be read as $O(N \log N)$
\end{solution}
\end{blocksection}