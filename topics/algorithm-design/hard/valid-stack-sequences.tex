\begin{blocksection}
\question Given two sequences pushed and popped with distinct values, return true if and only if this could have been the result of a sequence of push and pop operations on an initially empty stack. 



\begin{lstlisting}
class TwinListPriorityQueue<E implements Comparable> {
    ArrayList<E> L1, L2;
    void push(E item) {
        L1.push(elem);
        if (L1.size() >= Math.log(L2.size())) {
            L2.addAll(L1);
            mergeSort(L2);
            L1.clear();
        }
    }
    E pop() {
        E min1 = getMin(L1);
        E min2 = L2.poll();
        if (min1.compareTo(min2) < 0) {
            L1.remove(min1);
            return min1;
        } else {
            L2.remove(min2);
            return min2;
        }
    }
}
\end{lstlisting}

\begin{solution}
Let $N$ be the number of elements in the priority queue. Then the amortized
runtime for \lstinline$push$ is in $O(N)$ as the cost for every $\log N$
insertions is in $O(\log N \cdot 1 + 1 \cdot N \log N)$ which simplifies to
$O(N)$. Note that the size of \lstinline$L1$ is always constrained to be in
$O(\log N)$.

The amortized runtime for \lstinline$pop$ is also in $O(N)$. \lstinline$getMin$
on the unsorted list, \lstinline$L1$, is in $O(\log N)$, as with
\lstinline$L1.remove(min1)$. Polling from the front of \lstinline$L2$ is in
$\Theta(1)$. The most expensive component is \lstinline$L2.remove(min2)$ which
is in $O(N)$.
\end{solution}
\end{blocksection}
