\question Define a procedure, \texttt{multiples}, that returns an \texttt{SLList} containing the elements at indices \texttt{k}, \texttt{k + j}, \texttt{k + 2*j}, and so forth to the end of the list.

\ifprintanswers\else
\begin{lstlisting}
public class SLList {
    private IntNode sentinel;
    private static class IntNode {
        public int value;
        public IntNode next;
        public IntNode(int value, IntNode next) {
            this.value = value;
            this.next = next;
        }
    }
    public SLList() {
        this.sentinel = new IntNode(-1, null);
    }
    public SLList multiples(int k, int j) {















}

\end{lstlisting}
\fi

\begin{solution}
\begin{lstlisting}
public class SLList {
    private IntNode sentinel;
    private static class IntNode {
        public int value;
        public IntNode next;
        public IntNode(int value, IntNode next) {
            this.value = value;
            this.next = next;
        }
    }
    public SLList() {
        this.sentinel = new IntNode(-1, null);
    }
    public SLList multiples(int k, int j) {
        IntNode node = this.sentinel.next;
        for (int i = 0; i < k && node != null; i++) {
            node = node.next;
        }
        SLList multiples = new SLList();
        IntNode target = multiples.sentinel;
        while (node != null) {
            target.next = new IntNode(node.value, null);
            target = target.next;
            for (int i = 0; i < j && node != null; i++) {
                node = node.next;
            }
        }
        return multiples;
    }
}
\end{lstlisting}

\newpage
\begin{meta}
Whenever going through a Linked List problem, it's really helpful to show a concrete example! Try drawing out a decently sized example with different choices of $j,k$ and ask students questions before having them begin. 

Additionally, it could be useful to break this down to its pseudocode when explaining or giving a hint to stuedents. 

For example, 

1) Find the first node to start at (if it exists)

2) Create the SLList to return 

3) Add the first node and then add the nodes in indices $j$ steps at a time

4) Return the SLList
\end{meta}

\end{solution}

