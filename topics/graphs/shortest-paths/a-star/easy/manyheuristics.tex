%CHRISTIAN WU SPRING 2019
%REQUIRES Q/A ESPECIALLY ON SQRT PORTION

\question Let’s say you have a graph where the edges correspond to street blocks, nodes correspond to intersections, and edge-weights correspond to distance (in meters). Are the following heuristics valid? Are the following heuristics good?

\begin{enumerate}
    \item return euclidean distance between points
    \item return 0 for every pair of intersections
    \item return 1 for every pair of intersections
    \item return shortest distance through roads found by dijkstra's 
    \item return sqrt(euclidean distance) between points
    \item return number of lampposts between the intersections (found with a magical dataset stored in a hashmap)
\end{enumerate}


\begin{solution}

\begin{enumerate}
    \item This is admissible (distance is always going to be positive) 
    \\ This is consistent (0 < 0 + d(x, y)), 
    \\ This is not a very helpful heuristic (no estimation involved). 
    \\ This is Dijkstra's basically.
    \item This is not consistent (it overestimates distances less than 1).
    \item This is admissible (distance will always be at least the shortest possible distance)
    \\ This is consistent (d(a, x) <= d(a, x) + d(x, y))
    \\ Good heuristic : ) (this is the 'classic' heuristic)
    \item This works because it gives us perfect distances, but is horribly slow, as every update finds the shortest path. At this point we might as well just run Dijstra's.
    %DOUBLE CHECK THIS ONE
    \item This works! It is a monotonic function so it will keep the ordering.
    \item This is admissible because it will underestimate unless there are more than 1 lamppost per meter. (which is very unrealistic).
    \\ This is not consistent since the number of lampposts on a short road may be much larger than that on a long road.

\end{enumerate}

\end{solution}