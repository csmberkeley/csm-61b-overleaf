%ID: 612063
\marginpar{\subimport{../../linked-lists/naive/}{code.tex}
           \subimport{../../linked-lists/encapsulated/}{singly-linked.tex}
           \subimport{../../linked-lists/encapsulated/}{doubly-linked.tex}}
\question

Given a linked list of length $N$, give a tight asymptotic runtime
bound for each operation. Recall that \lstinline$IntList$ is a naive linked
list that only keeps track of the start node, \lstinline$SLList$ is an 
encapsulated singly-linked list with a front sentinel, and \lstinline$DLList$ 
is an encapsulated doubly-linked list with both front and back pointers.


\ifprintanswers\else
\renewcommand{\arraystretch}{2}
\setlength{\tabcolsep}{16pt}
\begin{tabularx}{\textwidth}{Xccc}
Operation                          & \lstinline$IntList$ & \lstinline$SLList$ & \lstinline$DLList$ \\ \hline
\lstinline$size()$                 &                     &                    &                    \\
\lstinline$get(int index)$         &                     &                    &                    \\
\lstinline$addFirst(E e)$          &                     &                    &                    \\
\lstinline$addLast(E e)$           &                     &                    &                    \\
\lstinline$addBefore(E e, Node n)$ &                     &                    &                    \\
\lstinline$addAfter(E e, Node n)$  &                     &                    &                    \\
\lstinline$remove(int index)$      &                     &                    &                    \\
\lstinline$remove(Node n)$         &                     &                    &                    \\
\end{tabularx}

Note: \lstinline{addBefore(E e, Node n)} inserts a node with element \lstinline{e} 
before node \lstinline{n} and \lstinline{addAfter(E e, Node n)} inserts a node with 
element \lstinline{e} after node \lstinline{n}.
You may assume that node \lstinline{n} already exists within the list.
\fi

\begin{solution}
\renewcommand{\arraystretch}{2}
\setlength{\tabcolsep}{16pt}
\begin{tabularx}{\textwidth}{Xccc}
Operation                          & \lstinline$IntList$    & \lstinline$SLList$     & \lstinline$DLList$     \\ \hline
\lstinline$size()$                 & $\color{red}\Theta(N)$ & $\color{red}\Theta(1)$ & $\color{red}\Theta(1)$ \\
\lstinline$get(int index)$         & $\color{red}O(N)$      & $\color{red}O(N)$      & $\color{red}O(N)$      \\
\lstinline$addFirst(E e)$          & $\color{red}\Theta(1)$ & $\color{red}\Theta(1)$ & $\color{red}\Theta(1)$ \\
\lstinline$addLast(E e)$           & $\color{red}\Theta(N)$ & $\color{red}\Theta(N)$ & $\color{red}\Theta(1)$ \\
\lstinline$addBefore(E e, Node n)$ & $\color{red}O(N)$      & $\color{red}O(N)$      & $\color{red}\Theta(1)$ \\
\lstinline$addAfter(E e, Node n)$  & $\color{red}\Theta(1)$ & $\color{red}\Theta(1)$ & $\color{red}\Theta(1)$ \\
\lstinline$remove(int index)$      & $\color{red}O(N)$      & $\color{red}O(N)$      & $\color{red}O(N)$      \\
\lstinline$remove(Node n)$         & $\color{red}O(N)$      & $\color{red}O(N)$      & $\color{red}\Theta(1)$ \\
\end{tabularx}

\lstinline$size()$: \lstinline$SLList$ and \lstinline$DLList$ both save size as
a field, whereas \lstinline$IntList$ has to iterate through the whole list to
find the size.

\lstinline$addBefore(E e, Node n)$: \lstinline$DLList$ has a pointer from each
node to the next node and to the previous node, so adding before is constant
time. \lstinline$IntList$ and \lstinline$SLList$ have to iterate through the
list from the start to find the node before the current node.

\lstinline$addAfter(E e, Node n)$: Regardless of which linked list structure we use,
we have access to node \lstinline{n}'s \lstinline{next} pointer.

\lstinline$remove(Node n)$: Because \lstinline$DLList$ has a pointer to the
previous node and next node, it can easily remove the node and point the
previous node to the next node (and point the next node point to the previous
node). However, \lstinline$IntList$ and \lstinline$SLList$ have to iterate
through the list to find the node before the current one.
\end{solution}

\pagebreak
\question
\begin{parts}
\part We add a method \lstinline{addAll(Collection<E> c)} to the \lstinline{IntList}, 
\lstinline{SLList}, and \lstinline{DLList} classes. Assuming no elements have been 
inserted before this method is called, and \lstinline{addAll}'s implementation repeatedly 
calls \lstinline{addLast}, give the runtime of \lstinline{addAll} in terms of $N$, the 
size of the collection \lstinline{c}.


\begin{solution}[1in]
\lstinline$IntList$: $\Theta(N^2)$ \\
\lstinline$SLList$: $\Theta(N^2)$ \\
\lstinline$DLList$: $\Theta(N)$

\lstinline$IntList$ and \lstinline$SLList$ both take $\Theta(N)$ time to
\lstinline$addLast(E e)$, where \lstinline$N$ is number of items currently in
the list. The trick here is to notice that we start adding to an empty list and
add until we fill the list to size \lstinline$N$. So every addition takes
variable time since the size of the list is different at each addition. We can
calculate the size of the list before each addition as follows:

$$0 + 1 + 2 + \cdots + N-1$$
$$= \frac12 [(0 + (N-1)) + (1 + (N-2)) + \cdots + ((N-2) + 1) + ((N-1) + 0)] = \frac12 (N(N-1))$$

Therefore, it takes $\Theta(N^2)$ time.
\end{solution}

\part Can we re-implement \lstinline{addAll} to improve its runtime? 
What is the new runtime for an \lstinline{IntList}, \lstinline{SLList}, and \lstinline{DLList}?

\begin{solution}[1in]
We can improve the runtimes of \lstinline{addAll} for an \lstinline{IntList}, 
\lstinline{SLList}.
Instead of calling \lstinline$addLast$, keep track of the last node and append
each new element in constant time to the end of the list. This improve the runtime
of \lstinline{addAll} for an \lstinline{IntList}, \lstinline{SLList} to $\Theta(N)$.
\end{solution}
\end{parts}
