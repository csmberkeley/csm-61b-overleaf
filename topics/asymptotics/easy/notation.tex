%ID: 612097
\begin{blocksection}
\question Define, in your own words, each of the following asymptotic notation.

\begin{parts}
\part $O$

\begin{solution}[0.25in]
`Big-O' notation gives an \emph{upper} bound on the runtime of a function as
the size of the input approaches infinity. It is not the same as a worst-case bound.
\end{solution}

\part $\Omega$

\begin{solution}[0.25in]
`Big-Omega' notation gives a \emph{lower} bound on the runtime of a function as
the size of the input approaches infinity. It is not the same as a best-case bound.
\end{solution}

\part $\Theta$

\begin{solution}[0.25in]
`Big-Theta' notation gives a \emph{tight} bound on the runtime of a function as
the size of the input approaches infinity. 

It is usually what is used to represent best and worst case bounds, but not always. If the runtime of a function is not guaranteed based on its input, this is when $O$ and $\Omega$ bounds might be needed to represent best and worst case bounds. 

For a more in-depth explanation of this, see Nadia's walkthrough. (Video being made)
\end{solution}

\end{parts}
\end{blocksection}
