\question Given the following function \lstinline$contrive$, categorize its runtime behavior by worst case, best case, and overall. 


\setlength{\tabcolsep}{12pt}
\begin{tabularx}{\textwidth}{XXXX}
         & O & $\Omega$ & $\Theta$\\\hline
Best-case    &    &    &\\
Worst-case    &    &   &\\
Overall        &    &    &\\
\end{tabularx}

\begin{lstlisting}

public void contrive(int n) {
    int N = 10000000;
    if (n == 0) {
        cubic(N);
    } else if (n % 2 == 0) {
        String coin_flip = flip();
        if (coin_flip.equals("Heads")) {
            logarithmic(N);
        } else if (coin_flip.equals("Tails")) {
            linear(N);
        }
    } else if (n % 2 == 1) {
        String coin_flip = flip();
        if (coin_flip.equals("Heads")) {
            quadratic(N);
        } else if (coin_flip.equals("Tails")) {
            exponential(N);
        }
    }
}

public String flip() {
    Random randomNum = new Random();
    int result = randomNum.nextInt(2);
    if (result == 0) {
        return "Heads";
    } else {
        return "Tails";
    }
}

void linear(int n) {
    int m = 0;
    for (int i = 0; i < n; i++) {
        m += i;
    }
    System.out.println(m);
}

void quadratic(int n) {
    int m = 0;
    for (int i = 0; i < n; i++) {
        for (int j = 0; j < n; j++) {
            m += j;
        }
    }
    System.out.println(m);
}

void cubic(int n) {
    int m = 0;
    for (int i = 0; i < n; i++) {
        for (int j = 0; j < n; j++) {
            for (int k = 0; k < n; k++) {
                m = i + j + k;
            }
        }
    }
    System.out.println(m);
}

void logarithmic(int n) {
    if (n <= 1) {
        return;
    } else {
        logarithmic(n / n);
        logarithmic(n / 2);
    }
}

int exponential(int n) {
    if (n == 1 || n == 2) {
        return 1;
    } else {
        return exponential(n - 1) + exponential(n - 2);
    }
}
\end{lstlisting}

\begin{solution}[1.5in]
\setlength{\tabcolsep}{12pt}
\begin{tabularx}{\textwidth}{XXXX}
         & O & $\Omega$ & $\Theta$\\\hline
Best-case  &  N &  log(N)  &  N/A  &\\
Worst-case & a^N  &  N^2  & N/A  &\\
Overall   &   a^N  & log(N)   &  N/A  &\\

\end{tabularx}
We can't write a tight bound when O and $\Omega$ are not equivalent. 
\end{solution}
