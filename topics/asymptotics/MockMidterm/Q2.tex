\question Under what conditions are the best case runtime achieved? Under what conditions are the worst case runtime achieved? Please give a tight overall runtime bound of the function using $\Omega$, $O$, and/or $\Theta$ notation in terms of $N$ and the size of the list \lstinline{x}, $M$.

% \begin{lstlisting}
% // x has nonnegative integers and the size is M 
% void t(int[] x; int N) {
%     boolean flag = true;
%     while (flag) {
% 	    flag = false;
% 	    for (int i = 0; i < x.length; i++) {
% 		    if(x[i] < N) {
% 			    x[i] += 1;
%                 flag = true;
%             }
%         }
%     }
% }
% \end{lstlisting}

\begin{lstlisting}
// x is an array of nonnegative integers of size M
void t(int[] x; int N) {
    boolean flag = true;
    while (flag) {
        flag = false;
        for (int i = 0; i < x.length; i++) {
            if (x[i] < N) {
                x[i] += 1;
                flag = true;
            }
        }
    }
}
\end{lstlisting}

\begin{solution}
The best case runtime is achieved if all of the elements have values greater than $N$. In this case, the program will loop once over all the elements in the array and the best case runtime is $\Theta(M)$

The worst case runtime is achieved if at least one element has a value of 0. This is because on each loop over the entire array, elements are only incremented by 1. This means that if a single element has a value of zero, the program will need to loop over the array $N$ times to increase the element in question to $N$. This means that the worst case runtime is $\Theta(MN)$. Note that this worst case runtime is met even if there is only a single 0 in the array.

Hence, the overall runtime is given by $\Omega(M)$, $O(MN)$.

\end{solution}