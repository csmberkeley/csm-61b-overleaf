\begin{lstlisting}
public static void f1(int N) {
    int sum = 0;
    for (int i = N; i > 0; i -= 1) {
        for (int j = 0; j < i; j += 1) {
            sum += 1;
        }
    }
}
\end{lstlisting}
\begin{solution}
\Theta(N^2)
\end{solution}

\begin{lstlisting}
public static void f2(int N) {
    int sum = 0;
    for (int i = 1; i <= N; i *= 2) {
        for (int j = 0; j < i; j += 1) {
            sum += 1;
        }
    }
}
\end{lstlisting}
\begin{solution}
\Theta(N)
\end{solution}


\begin{lstlisting}
public static void f3(int[] a) {
    if (a.length == 0) { return; }
    int N = a.length;
    int[] newA = new int[N-1];
    for (int i = 0; i < newA.length; i += 1) {
        newA[i] = a[i];
    }
    f3(newA);
}
\end{lstlisting}
\begin{solution}
\Theta(N^2)
\end{solution}

\begin{lstlisting}
public static void f5(int N) {
    f1(N);
    f2(N);
    f3(new int[N]);
    f4(N); // \Theta(N)
}
\end{lstlisting}
\begin{solution}
\Theta(N^2)
\end{solution}

\begin{lstlisting}
public static void spacejam(int N) {
    if (N == 1) {
        return;
    }
    
    for (int i = 0; i < N; i += 1) {
        spacejam(N - 1);
    }
}
\end{lstlisting}
\begin{solution}
\Theta(N!)
\end{solution}

\begin{lstlisting}
public static void f(int N) {
    f1(N);
    f2(N);
    f3(new int[N]);
    f4(N); // \Theta(N)
}
\end{lstlisting}
\begin{solution}
\Theta(N^2)
\end{solution}

\begin{lstlisting}
public static void g1(int N) {
	for (int i = 0; i < N*N*N; i += 1) {
		for (int j = 0; j < N*N*N; j += 1) {
			System.out.print("fyhe");
		}
	}
}
\end{lstlisting}
\begin{solution}
\Theta(N^6)
\end{solution}

\begin{lstlisting}
public static void g2(int N) {
	for (int i = 0; i < N; i += 1) {
		int numJ = Math.pow(2, i + 1) -1; // <--constant time!
		for (int j = 0; j < numJ; j += 1) {
			System.out.print("fhet");
		}
	}
}
\end{lstlisting}
\begin{solution}
\Theta(2^N)
\end{solution}

\begin{lstlisting}
public static void g3(int N) {
	for (int i = 2; i < N; i *= i) {}
	for (int i = 2; i < N; i++) {}
}
\end{lstlisting}
\begin{solution}
\Theta(N)
\end{solution}

\begin{lstlisting}
public static void g4(int N) {
	if (N == 0) { 
		return; 
	}
	g4(N -1);
	if (k(N)) { 
		g4(N -1); 
	}
}
\end{lstlisting}
\begin{solution}
\Theta(N)
\end{solution}

$O(2^N)$
\begin{lstlisting}
public static void g5(int N) {
	if (N == 0) { 
		return; 
	}
	g5(N / 2);
	if (k(N)) { 
		g5(N / 2); 
	}
}
\end{lstlisting}
$O(log(N))$
$O(N)$