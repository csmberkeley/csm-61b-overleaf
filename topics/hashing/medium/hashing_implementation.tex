\begin{blocksection} 

You are a software engineer for a ticket-selling company who wishes to find a more efficient way to store customer information than using a simple array, and you decide to store info using hashing with singly-linked lists. Design a put method for TicketTable to work properly. Assume a load factor of 3.0.
\\

\begin{lstlisting}
public class Customer {
    public Customer tail;
    public int ticketID;
    private string name;
    private boolean VIP;
    
    public int hashCode() {
        return ticketID;
    }
    
public class TicketTable {
    private Customer[] heads;
    private int bucketN, size; // # of buckets in the table, # of customers.

    private void resize() { // resizes the table to have more buckets, resorts all customers.
        ...
    }
    
    //initializers and other methods not shown
}
\end{lstlisting}
\end{blocksection}
\clearpage

\question

\begin{lstlisting}
public void put(Customer fred) {

    if (_________________________________) {
    
        _______________________________________________;
    }
    ____________________________ = ________________________________;
    
    if (_________________________________) {
    
    ____________________________ = ________________________________;
    } else {
        while (____________________________________) {
        
            _______________________ = _____________________________;
        }
        _______________________ = _____________________________;
    }
    size++;
}
\end{lstlisting}

\begin{solution}
\begin{lstlisting}
public void put(Customer fred) {
    if ((size + 1)/bucketN >= 3) {
        resize();
    }
    Customer c = heads[fred.hashcode() % bucketN];
    if (c == null) {
        heads[fred.hashcode() % bucketN] = fred;
    } else {
        while (c.tail != null) {
            c = c.tail
        }
        c.tail = fred;
    }
    size++;
}
\end{lstlisting}
\end{solution}