%ID: 612062
\question Here’s a class for a hashtable that takes key, value pairs and stores them in a hashtable with external chaining. The keys in this problem are all strings, and the hashcode for each string is its length. Implement the hash function \lstinline{hashfunc(String key)} and complete the method \lstinline{insert} that inserts new key value pairs into the hashtable. You will be implementing the part of \lstinline{insert} that handles resizing. It should be done recursively.

\begin{lstlisting}
// A node of the external chains
class HashNode<String, V> { 
    String key; 
    V value; 
    HashNode<String, V> next; 
  
    public HashNode(String key, V value) { 
        this.key = key; 
        this.value = value; 
    } 
} 

// class for the actual table
class HashTable<String, V> {
    private HashNode<String,V>[] hashtable;
    private int numBuckets;
    private int numElements;
    private int maxLoad = 0.75;

    public HashTable() { 
        numBuckets = 10; 
        numElements = 0;
        hashtable = new HashNode<String,V>[numBuckets]; 
  
    %     // Create empty chains 
    %     for (int i = 0; i < numBuckets; i++) 
    %         hashtable[i] = null; 
    % } 

    // Given a key, returns the hashcode for that key
    private int hashfunc(String key) {
        return __________________;
    }

    public void insert(String key, V value) {
        int index = hashfunc(key);
        HashNode<String, V> head = hashtable[index];

        //check if key is already in table
        while (head != null) {
            if (head.key.equals(key)) { 
                head.value = value; 
                return; 
            } 
            head = head.next;
        }

        // resize
        if (_______________________) {
            numBuckets = ___________________;
            HashNode<String,V>[] temp = hashtable;
            hashtable = new HashNode<String,V>[numBuckets];
            numElements = _____________;

            for (int i = 0; i < numBuckets; i++) {
                ________________________________; 
  	}

            for (HashNode<String, V> headNode : __________) { 
                while (_________________________) {
                    ________________________________________;
                    ________________________________________;
                } 
            }
        }

        // add new key, value as a HashNode
        numElements++;
        index = hashfunc(key);
        head = hashtable[index];
        HashNode<String, V> newNode = new HashNode<String, V>(key, value);
        newNode.next = head;
        hashtable[index] = newNode;
    }
}
\end{lstlisting}

\begin{solution}
\begin{lstlisting}
    private int hashfunc(K key) {
        return key.length() % numBuckets;
    }
    
    // resize
    // Note we cast numElements+1 to a float so integer division
    // does not give us a floored integer
    if ((double) (numElements+1) / numBuckets > maxLoad) {
        numBuckets *= 2;
        HashNode<K,V>[] temp = hashtable;
        hashtable = new HashNode<K,V>[numBuckets];
        numElements = 0;

        for (int i = 0; i < numBuckets; i++) 
            hashtable[i] = null;

        for (HashNode<K, V> headNode : temp) { 
            while (headNode != null) { 
                insert(headNode.key, headNode.value); 
                headNode = headNode.next; 
            } 
        }
    }


\end{lstlisting}
\end{solution}