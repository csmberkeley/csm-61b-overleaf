\question
\begin{lstlisting}
class Dog {
    public String className;
    public Dog() {
        className = "dog";
    }
    public String getClassName() {
        return className;
    }
}
class Beagle extends Dog {
    public String className;
    public Beagle() {
        super();
        className = "beagle";
    }
}
class Chihuahua extends Dog {
    public String className;
    public Chihuahua() {
        super();
        className = "chihuahua";
    }
    @Override
    public String getClassName() {
        return className;
    }
}
\end{lstlisting}

\begin{solution}
Note that this behavior, \textit{field hiding}, is \textbf{not a part of the course}. You will not be expected to know how to solve these problems on an exam.
\end{solution}

What would Java display?

\begin{parts}
\part
\begin{verbatim}
Dog d = new Chihuahua();
System.out.println(d.getClassName());
\end{verbatim}
\begin{solution}[0.75in]
\texttt{chihuahua}. \texttt{d.getClassName} is an instance method so we look at its dynamic type, \texttt{Chihuahua}.
\texttt{getClassName} returns \texttt{d}'s \texttt{Chihuahua.className}, \texttt{chihuahua}.
\end{solution}

\part
\begin{verbatim}
Dog d = new Beagle();
System.out.println(d.className);
\end{verbatim}
\begin{solution}[0.75in]
\texttt{dog}. \texttt{d.className} is a field so we use its static type.
\texttt{d}'s \texttt{Dog.className} is \texttt{dog}.
\end{solution}

\part
\begin{verbatim}
Beagle d = new Beagle();
System.out.println(d.getClassName());
\end{verbatim}
\begin{solution}[0.75in]
\texttt{dog}. \texttt{d.getClassName} is an instance method so we use its dynamic type: \texttt{Beagle}.
\texttt{Beagle} inherits \texttt{getClassName} from \texttt{Dog} so the static type of \texttt{this} in \texttt{Dog.getClassName} is \texttt{Dog}. As a result, we return \texttt{d}'s \texttt{Dog.className}, \texttt{dog}.
\end{solution}
\end{parts}