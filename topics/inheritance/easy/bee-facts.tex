\question
\begin{lstlisting}[aboveskip=-\baselineskip]


public interface Bee {
    public void nom();
}
public abstract class HoneyBee implements Bee {
    String name;
    String sex;
    public abstract void dance();
    public void sting() {
        if (sex.equals("female")) {
            System.out.println(name + " stings her target! Ouch!");
        }
        else {
            System.out.println(name + " is a drone (male bee)! He doesn't have a stinger :(.");
        }
        
    }
}
public class GuardBee extends HoneyBee {
    public void dance() {
        System.out.println("Guard bee " + name + " does the 'I see you' (ISY) dance to warn workers of a nearby threat, and to let the potential attacker know that their plan has been exposed!");
    }
    public void nom() {
        System.out.println("Guard bee " + name + " chows down on the worker bee's vomit (honey)! Yum!");
    }
}
\end{lstlisting}

\begin{parts}
\begin{blocksection}
\part Compare and contrast interfaces and abstract classes.
\begin{solution}[1in]
\begin{itemize}
\item Java classes cannot extend multiple superclasses (unlike Python) but
classes can implement multiple interfaces.
\item Interfaces are implicitly public.
\item Interfaces can't have fields declared as instance variables; any fields
that are declared are implicitly \lstinline$static$ and \lstinline$final$.
\item Interfaces use the \lstinline$default$ keyword to declare \emph{concrete}
implementations while abstract classes use the \lstinline$abstract$ keyword to
declare \emph{abstract} implementations.
\item Interfaces define the way we interact with an implementing object or
functions of an object. Conversely, abstract classes define an ``is-a''
relationship and tell us more about the object's fundamental identity.
\end{itemize}

\begin{meta}

\textbf{Meta:} Warn students to be careful searching about this topic as CS 61B
uses Java 8 but most content online covers Java 7, which behaves differently.
\end{meta}

\end{solution}
\end{blocksection}

\begin{blocksection}
\part Do we need the \lstinline$sting$ method in \lstinline$GuardBee$?

\begin{solution}[0.5in]
No, we do not need the \lstinline$sting$ method because it's already defined in
the abstract class. Java will lookup the parent class's method if it cannot
find it in the child class.
\end{solution}
\end{blocksection}

\begin{blocksection}
\part Does this compile? \lstinline$Bee bweep = new GuardBee();$

\begin{solution}[0.5in]
Yes, the code compiles since \lstinline$GuardBee$ inherits from the
\lstinline$HoneyBee$ class which implements the \lstinline$Bee$ interface.
\end{solution}
\end{blocksection}
\end{parts}

\begin{solution}
%\textbf{Meta:} Use this question as a mini-lecture for interfaces to build up
%to the next question. This question should not take a lot of time.
\end{solution}
