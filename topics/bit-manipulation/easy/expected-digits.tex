% Author: Ashley Ye Spring 2022

\question Given a decimal integer and an expected number of digits, manipulate the binary representation of that integer to have the expected number of digits (with no leading zeros) using bit operations.

Return the manipulated number in decimal after the operations.

For instance, 21 can be represented in binary form as 10101.

expectedDigits(21, 5) would return 21. \newline 
expectedDigits(21, 6) would return 42. \newline
expectedDigits(21, 4) would return 10. 

\ifprintanswers\else
\begin{lstlisting}
static int expectedDigits(int num, int expected) {

   int upperLimit = _____________________________________________
   int lowerLimit = _____________________________________________
   
   while (_____________________________________________) {
     _____________________________________________
   }
 
   while (_____________________________________________) {
     _____________________________________________
   }
   
   return num;
}

\end{lstlisting}
\fi

\begin{solution}
\begin{lstlisting}

static int expectedDigits(int num, int expected) {

   int upperLimit = 1 << expected;
   int lowerLimit = 1 << (expected - 1);
   
   while (num >= upperLimit) {
     num = num >> 1;
   }
 
   while (num < lowerLimit) {
     num = num << 1;
   }
   
   return num;
}

// Given decimal integer num, manipulating its binary representation to have expected number of digits is to divide or multiply num by 2, i.e. left/right shift num to get a binary number with the expected number of digits. 

// If num is larger than the largest binary number with expected digits, i.e. 2^(expected digits), then we left shift/ divide by 2 to reduce num's bit digits.

// If num is smaller than the smallest binary number with expected digits, i.e. 2^(expected digits - 1), then we right shift/ multiply by 2 to increase num's bit digits.

// Otherwise, we return num itself. 


\end{lstlisting}
\end{solution}