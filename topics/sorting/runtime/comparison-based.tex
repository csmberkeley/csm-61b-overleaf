\ifprintanswers\else
{
\renewcommand{\arraystretch}{2}
\setlength{\tabcolsep}{12pt}
\begin{tabularx}{\textwidth}{Xlll}
Algorithm         & Best-case & Worst-case & Stable \\\hline
Selection Sort    &           &            &        \\
Insertion Sort    &           &            &        \\
Merge Sort        &           &            &        \\
Heapsort          &           &            &        \\
Quicksort         &           &            &
\end{tabularx}
}
\fi

\begin{solution}
{
\renewcommand{\arraystretch}{2}
\setlength{\tabcolsep}{12pt}
\begin{tabularx}{\textwidth}{Xlll}
Algorithm         & Best-case          & Worst-case          & Stable \\\hline
Selection Sort    & $\Theta(N^2)$      & $\Theta(N^2)$       & Depends\\
Insertion Sort    & $\Theta(N)$        & $\Theta(N^2)$       & Yes    \\
Merge Sort        & $\Theta(N \log N)$ & $\Theta(N \log N)$  & Yes    \\
Heapsort          & $\Theta(N)$        & $\Theta(N \log N)$  & Hard	  \\
Quicksort         & $\Theta(N)$        & $\Theta(N^2)$       & Depends
\end{tabularx}
}

\begin{description}
\item[Selection Sort]
In selection sort, we loop through the array to find the smallest element.
Next, we swap the element at index-0 with the smallest element. Next, we repeat
this procedure, but only looking at the array starting at index-1.

\begin{itemize}
\item \textit{Runtime, Best, Worst Case:} Since it takes $O(N)$ time to loop
through the array, and we loop through the array $N$ times, this algorithm has
a runtime of $\Theta(N^2)$. Note that even if the array is already sorted, we
need to iterate through it to find the minimum, and then iterate through it
again, and again, $N$ times.
\item \textit{Stability:} Consider an array \lstinline${ 3A, 2, 3B, 1 }$, where
the $3$s have been labeled to differentiate between them. The algorithm will
find $1$ to be the smallest, and will swap it with $3A$, pushing $3A$ after
$3B$, making it not stable. However, it is also possible to make it stable if
we implement Selection Sort in a different way, which involves creating a new
array instead of swapping the minimum elements.
\end{itemize}

\item[Insertion Sort]
This is the way an adult would normally sort a pack of cards. Iterating through
the array, swapping each element left-wards.

\begin{itemize}
\item \textit{Best Case:} Given a sorted array, \lstinline${ 1, 2, 3, 4 }$,
this algorithm would iterate through the array just once, and do 0 swaps, since
all elements are already as left-wards as they can be.
\item \textit{Worst Case:} Given a fully unsorted array,
\lstinline${ 4, 3, 2, 1 }$, this algorithm would first swap $(3, 4)$, then to
move 2 left-wards, it needs to do 2 swaps. Finally to move 1 left-wards, it
needs to do 3 swaps. This is of the ordering of $O(n^2)$ swaps.
\item \textit{Stability:} Consider an array \lstinline${ 3A, 2, 3B, 1 }$. We
would get the following steps: \lstinline${ 2, 3A, 3B, 1 }$,
\lstinline${ 1, 3, 3A, 3B, }$. In general, this algorithm is stable, because
given a $3A, 3B$, we would never swap them with each other.
\end{itemize}

\item[Merge Sort]
Given an array, divide it into two equal halves, and call mergesort recursively
on each half. Take the recursive leap of faith and assume that each half is now
sorted. Merge the two sorted halves. Merging takes a single iteration through
both arrays, and takes $O(N)$ time. The base case is if the input list is just
1 element long, in which case, we return the list itself.

\begin{itemize}
\item \textit{Best case, Worst Case, Runtime:} Since the algorithm divides the
array and recurses down, this takes $\Theta(N \log N)$ time, no matter what.
\item \textit{Stability:} Merge sort is made stable by being careful during the
merging step of the algorithm. If deciding between 2 elements that are the
same, one in the left half and one in the right half, pick the one from the
left half first.
\end{itemize}

\item[Heap Sort]
Place all elements into a heap. Remove elements one by one from the heap, and
place them in an array.

\begin{itemize}
\item \textit{Recall:} Creating a heap of $N$ elements takes $N \log N$ time,
because we have to bubble-up elements. Removing an element from a heap takes
$\log N$ time, also because of bubbling and sinking.
\item \textit{Best Case:} Say that all the elements in the input array are
equal. In this case, creating the heap only takes $O(N)$ time, since there is
no bubbling-down to be done. Also, removing from the heap takes constant time
for the same reason. Since we remove $N$ elements, and creating the heap takes
$O(N)$ time, the overall runtime is $O(N)$.
\item \textit{Worst Case:} Any general array would require creating the heap
with bubbling which itself takes $N \log N$ time.
\end{itemize}

\item[Quicksort]
Based on some pivot-picking strategy, pick a pivot.  Divide the array up into 3
groups: elements smaller than the pivot, larger than the pivot and equal to the
pivot. Recursively sort the first and second group.

\begin{itemize}
\item \textit{Runtime:} Analyzed in detail in the next question.
\end{itemize}
\end{description}
\end{solution}
