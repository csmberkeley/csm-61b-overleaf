\begin{blocksection}
\question If the worst-case runtime of Quicksort is $O(N^2)$ while the
worst-case runtime of Mergesort is $O(N \log N)$, why do we ever use Quicksort?

\begin{solution}[1.5in]
In most cases, Quicksort actually performs very well, with an average runtime
of $O(N \log N)$. The probability that Quicksort ends up running in $O(N^2)$
is, in reality, very, very low. There is a mathematical discussion about this
probability \href{https://www.khanacademy.org/computing/computer-science/algorithms/quick-sort/a/analysis-of-quicksort}{here}.

Most implementations of MergeSort requires extra space (more arrays to be
created). However, there are more complicated ways to write a MergeSort
algorithm that doesn't require any extra space, but Quicksort is still often
preferred when the number of elements being sorted isn't extremely large.

The general idea for this is that if all the data being sorted fits inside your
RAM, then Quicksort is faster, since it doesn't need to access data stored in
your Harddisk, which is slower. For reference, in 1GB RAM, we can hold an array
of 32 million integers. This means, that for almost all regular purposes,
Quicksort is faster. More information about this is available on the first and
second answer \href{https://stackoverflow.com/questions/70402/why-is-quicksort-better-than-mergesort}{here}.
\end{solution}
\end{blocksection}
