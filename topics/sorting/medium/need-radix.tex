\begin{blocksection}
\question Performing Radix Sort seems to be a fast sorting algorithm. Why don't
we always use it?

\begin{solution}[1in]
Radix sort is only possible when the elements being compared have some radix or
base. For strings, they can be broken up into individual characters and
separately compared, as we did above. We can also do the same for integers.
However, what if we wanted to compare 10 \lstinline$Cat$ objects? We cannot
compare \lstinline$Cat$ objects by breaking them up into any smaller pieces or
radices.\\
Radix sort also takes a lot more memory than all of our other choices. This is because we need at the very least $N$ extra memory to store the new ordering if we are only trying to find the memory in terms of $N$. 
\end{solution}
\end{blocksection}
